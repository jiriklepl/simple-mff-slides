%\documentclass{beamer}
% because the year is likely >2015, might want to support widescreen:
\documentclass[aspectratio=169]{beamer}
% (the default is 43, i.e. 4:3)


\usepackage[T1]{fontenc}
\usepackage[utf8]{inputenc}
\usepackage{lmodern}
\usepackage{xcolor}

\usepackage{tikz}\usetikzlibrary{positioning} %remove if not using diagrams
\usepackage{listings} %remove this if you don't need code listings
\usepackage{fontawesome5} %remove this if you don't use "good/bad" lists and nice icons

% Styling. Highly recommend using the metropolis theme, which has many small
% `features' that help a lot with readability and comprehensibility.
\usetheme{metropolis}

% Careful with the colors!
%\setbeamercolor{normal text}{fg=black!75,bg=black!2}
%\setbeamercolor{alerted text}{fg=green!80!black}
%\setbeamercolor{palette primary}{fg=red!10,bg=violet!50!black}
%\setbeamercolor{palette primary}{bg=red!40!black}
%\setbeamercolor{palette primary}{bg=green!50!yellow!80!black}

\usepackage{FiraSans}
% some other font packages:
%\usepackage[defaultsans]{droidsans}
%\usepackage[sfdefault]{cabin}
%\usepackage[math]{iwona}
%\usepackage[sfdefault,light]{roboto}

\author{Jiří Klepl}
\title{Inference-driven resource managemenent and polymorphism in systems programming} %really change this!
\date{MFF UK, September 2022}

\newcommand{\xxx}[1]{\textcolor{red!}{#1}}

\begin{document}
\maketitle

% Start with roughly 3 slides that work as an `abstract'. Give a totally brief
% overview of what is the motivation for the thesis, how you approached it, and
% what is the main result.
%
% The point of this is that your problem, approach and results, if coherent,
% will deliver a message to the commitee that your thesis MAKES SENSE. The
% sooner they know, the better.
%
% I follow with a demonstration that shows how I would present and defend why I
% developed this `template' slideshow.

\begin{frame}{Goals}
  \begin{itemize}
    \item Reduction of code repetition in systems programming
    \item Type inference of variables and record fields
    \item Automatic resource management
    \item Tackling subtypes
      \begin{itemize}
        \item Many low-level languages have various \emph{constness} requirements (\lstinline{constexpr}) on variables or some other requirements that are parallel to the types recognized by the languages
      \end{itemize}
  \end{itemize}
  \pause
  How do we achieve these goals? \xxx{do a picture for each}
  \pause
  \begin{itemize}
    \item Parametric polymorphism (idea: ``linear'' type functions)
    \item Multi-parameter typeclasses (overloading constrained by a relation)
    \item Automatic calls to resource management functions (bound to a resource lifetime)
    \item New extension to Hindley-Milner-style type systems
  \end{itemize}
\end{frame}

\begin{frame}{Proof-of-concept compiler}
  \begin{enumerate}
    \item Tokenization and parsing (\xxx{arrow} AST)
    \item Flow and variable live-range analysis (\xxx{arrow} annotated AST and \xxx{some metadata})
    \begin{itemize}
      \color{olive}
      \item Automatic resource management added to the AST
    \end{itemize}
    \item Constraint generation and AST elaboration
    \item Type inference
    \item Monomorphization and postprocessing of the elaborated AST according to the inferred types (name mangling, insertion of inferred types)
    \item Assembly generation
  \end{enumerate}
\end{frame}

\begin{frame}{\xxx{Focus: type inference}}
  \begin{itemize}
    \item Parametric polymorphism (idea: ``linear'' type functions)
    \item Multi-parameter typeclasses (overloading constrained by a relation)
    \item Automatic calls to resource management functions (bound to a resource lifetime)
    \item New extension to Hindley-Milner-style type systems
  \end{itemize}
\end{frame}

\begin{frame}[standout]{Je to na githubu + vyzkouset}
  \url{https://github.com/jiriklepl/masters-thesis-code}
\end{frame}

\begin{frame}{proc}
  Systems programming covers wide range of applications:

  \begin{itemize}
    \item Operating systems and drivers
    \item real-time mission-critical systems
    \item embedded systems
  \end{itemize}

  Common characteristics:

  \begin{itemize}
    \item Strict requirements on resource usage: power, time, performance, efficiency, etc.
    \item Reliability and errorlessness
    \item Maintainability
    \item \xxx{go on}
  \end{itemize}
\end{frame}

\begin{frame}{Our starting point}
  The previous work showed possible benefits of combining a procedural language and a Hindley-Milner-style type system, it lacked: (\xxx{make it a bad color})

  \begin{itemize}
    \item Type inference of record fields
    \begin{itemize}
      \item Multi-parameter typeclasses with functional dependencies (type functions)
    \end{itemize}
    \item Support for automatic resource management
    \item Type system coverage of subtypes (C \lstinline{const}, \lstinline{volatile})
    \item Extensibility with more advanced type features (e.g. existentials)
  \end{itemize}
\end{frame}

\begin{frame}{rozdil c-{}- a c}
\end{frame}

\begin{frame}{Multi-parameter type classes + proc}
  \begin{itemize}
    \item 
  \end{itemize}
\end{frame}

\begin{frame}{proc to je slozity}
\end{frame}

\begin{frame}{detaily implementace (asi 4 slajdy)}
\end{frame}

\begin{frame}{najschlee}
\end{frame}

% if your approach idea can be made this simple, use a standout slide
\begin{frame}[standout]{Main approach idea}
Make example slides
$\to$
Put them on GitHub
\end{frame}

\begin{frame}{Results so far}
The impact is not known, because I am uploading the slides today.

\pause
From the availabe data (from other similar efforts), we can assume that:
\begin{itemize}
% some extra beamer magic
\item \alert<2>{More than 1 student will download and use the template}
\item \alert<3>{More than 5 minutes of time will be saved.} \uncover<3>{Likely more.}
\end{itemize}
\end{frame}

\section{Making the slides}
% This is basically to say that you start talking about "details", name it as
% you need it for your presentation

\begin{frame}{Contents}
The template is a normal metropolis-themed beamer document.

Main focus:
\begin{itemize}
\item Focus on fast dive-in \pause
  \begin{itemize}
  \item Decreases the risk of not getting to the results
  \item This is what presentations should look like, right? \pause
  \end{itemize}
\item Demo the common beamer tricks (pause, standout, \dots)
\end{itemize}
\end{frame}

\begin{frame}[standout]{Remember!}
Any slides may be improved by removing 50\% text \\ and adding 100\% more pictures!
\end{frame}

\begin{frame}{Demo: TikZ diagram}
\centering
\tikzstyle{rec}=[rectangle, draw, rounded corners=1ex, font=\huge\bfseries]
\begin{tikzpicture}[ultra thick, inner sep=1ex]
\node[rec] (a) {Keep};
\node[rec, circle, right=of a] (b) {it};
\node[rec, right=of b] (c) {simple};
\node[rec, densely dotted, below=2cm of c, font=\small] (notice) {\dots{}but precise!};
\draw[->] (a) to (b);
\draw[->] (b) to (c);
\draw[dotted] (notice) to (c);
\end{tikzpicture}
\end{frame}

\begin{frame}[fragile]{Demo: including formatted source}
% the fragile modifier (above) may be needed with many kinds of verbatim
% environments, also with lstlistings etc.

\begin{lstlisting}[language=C,showstringspaces=false,basicstyle=\tt\small,commentstyle=\color{green!50!black},keywordstyle=\bfseries\color{blue!50!black},stringstyle=\color{red!50!black}]
int main() {
  printf("The answer is: %d\n", 6*7); //actually 6*9
  return 0; //no chance this didn't succeed
}
\end{lstlisting}
\end{frame}

\begin{frame}{Demo: showing a picture with detailed results}
\centering
\includegraphics[height=0.8\textheight]{img/ukazka-obr01.pdf}
\end{frame}

\begin{frame}{Demo: showing something with comments}
\begin{columns}
\begin{column}{0.5\textwidth}
\includegraphics[width=\linewidth]{img/ukazka-obr01.pdf}
\end{column}
\begin{column}{0.5\textwidth}
How we arrived at this?
\begin{enumerate}
\item Generated the data
\item Plotted them
\item Used a very fine red marker to circle the points
\end{enumerate}

`Advantages' demo:

\begin{itemize}
\item[\color{green}\faCheck] Data is normal
\item[\color{red}\faTimes] Data is sparse
\item[\color{violet}\faQuestionCircle] What now?
\end{itemize}
\end{column}
\end{columns}
\end{frame}

% Ending frame is important -- in the rare case that you actually manage to
% finish the presentation on time, it avoids the embarassing seconds of silence
% between the time when you finish and the moment when committee detects that
% something is wrong and they should continue.
\begin{frame}[plain]
\centering
{\Large\bfseries Thank you for attention!}

% You can try the results on GitHub: \\
% \url{https://github.com/exaexa/simple-mff-slides}

\end{frame}
\end{document}
